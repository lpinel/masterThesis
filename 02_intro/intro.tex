\chapter{Introduction}
Industry was one of the first fields of application of robotics, where the environment is mainly static, the tasks to be performed are repetitive and automated, and the human interaction is quite low. For that reason, the idea of designing robots able to work in dynamic environments, with a high variety of tasks and interacting with humans and their environment, was fulfilled thanks to the evolution of new techniques related to robotics. Humanoid robots, physically similar to the human being, meet all that needs. Mainly, the possibility of moving, solves the problem of industrial robots that can only work in fixed areas. Moreover, the provision of artificial intelligence, allows the robot to interact with the surrounding environment in a more natural way, as the human being.
%Uno de los primeros campos de aplicación de la robótica fue la industria, donde el entorno del robot es prácticamente estático, las tareas que realiza son repetitivas y automatizadas, y la interacción con el ser humano es relativamente baja. Por ello surgió la necesidad de diseñar robots capaces de trabajar en entornos cambiantes, que sean capaces de realizar tareas de muy diversa índole y que la interactuación con el entorno y el ser humano sea plena. Los robots humanoides, con un aspecto físico similar al del ser humnano, resuelven todas estas últimas necesidades. La posibilidad de desplazarse, a diferencia de los robots industriales, soluciona la problemática de trabajar en un espacio de trabajo fijo. Además, la dotación de inteligencia artificial, permite al robot interactuar con el entorno que lo rodea de una forma mucho más natural para el ser humano.\\

Nevertheless, the possibility of moving, brings the problem of stability. Maintaining the humanoid robot in an upright posture and walking is a complex task related to control. For humans, walking is so simple that we do almost unconsciously, so we are not aware of its complexity. At all the times, it had to be ensured that the robot is in an upright posture in order to not to fall over and simultaneously, it is performing a series of movements previously defined to walk. As well, in a unbalanced situation, humans, unconsciously, try to stabilize moving their own body or the other limbs and the same behaviour is expected for the humanoid robot.

%Sin embargo, la posibilidad de desplazarse trae consigo una gran problemática, la estabilidad. El hecho de que el robot se mantenga erguido y camine, es una tarea complicada desde el punto de vista del control. Sin embargo, para el ser humano, el caminar es una tarea que realiza casi inconsicientemente, por lo que no es consciente de su complejidad. En todo momento, se debe asegurar que el robot se mantenga lo más erguido posible para no caer mientras simultáneamente está realizando una serie de movimientos predefinidos para caminar. Asímismo, ante una situación de desequilibrio, un ser humano, involuntariamente, intenta estabilizarse moviendo el resto de extremidades, lo que es un complicado comportamiento para implementar en un robot.\\

First works about biped robots were carried out about 1970 by authors Kato \cite{Kaj2005} and Vukobratović. The first anthropomorphic robot, WABOT-1, was exhibited by Kato in 1973 in Waseda University (Japan). Using a very simple control diagram, the robot was able to perform a few slow gaits, maintaining its balance all the times. This achievement, was the first one that encouraged researchers about humanoid robots and their locomotion.

%Los primeros trabajos en cuanto a robots bípedos fueron llevados a cabo sobre 1970 por los autores Kato \cite{Kaj} y Vukobratović. El primer robot antropomórfico, WABOT-1, fue exhibido por Kato en 1973 en la Universidad de Waseda (Japón). Usando un esquema de control bastante sencillo, el robot esra capaz de realizar unos pocos pasos lentos, manteniéndose estable en todo momento. Éste logro, fue el primero que desencadenó la investigación acerca de robots humanoides y de su locomoción.\\

At the same time, Vukobratović and his research team were studying stability in biped systems in the former Yugoslavia, basing on a new stability criterion, presented in 1972, as \textit{Zero-Moment Point (ZMP)}. Taking into account the dynamic effects produced during a walking, from then until now, the ZMP stability criterion has been the most used in humanoid or biped robotics.
%Paralelamente, Vukobratović y su equipo investigaban en estabilidad para sistemas bípedos en Yugoslavia, basándose en un nuevo criterio de estabilidad, presentado en 1972, como \textit{Zero-Moment Point (ZMP)}. Teniendo en consideración los efectos dinámicos que se producen durante la caminata, desde entonces hasta la actualidad, el criterio de estabilidad del ZMP ha sido el más utilizado en cuanto a robótica humanoide o bípeda se refiere.\\

The rise of humanoid robotics started with the development of P2 robot by the company Honda in 1996 \cite{Kaj2005}. The project began in secret ten years before, after the exhibition of WABOT-2 playing the piano. P2 (180 centimetres high and 210 kg weight), was the firs humanoid able to walk in a stable enough way and carry its processor and battery on its back. After that, robots P3 and ASIMO were its advanced versions, reducing hight and weight of the robot.

%El auge de la robótica humanoide comenzó con el desarrollo del robot P2 por la empresa Honda en 1996 \cite{Kaj} . El proyecto comenzó en secreto el proyecto diez años antes, tras el lanzamiento del robot WABOT-2 tocando el piano. P2, de 180 centímetros de altura y 210kg de peso, fue el primer humanoide que podía caminar de forma suficientemente estable y cargar con el procesador y la batería a la espalda. Tras éste, los robots P3 y ASIMO fueron sus versiones avanzadas, reduciendo la estatura y el peso del robot.

\section{Objectives}
This Master Thesis deals with the balance control of humanoid robot TEO (\textit{Task Environment Operator}) using Force-Torque sensors. The design of the control system must include the sensors feedback in real time to maintain stability.

The main objectives of this work are:
\begin{itemize}
\item Data acquisition in real time form Force-Torque sensors assembled in the robot ankles, which are the main feedback of the control loop.
\item Computation of the ZMP \textit{Zero-Moment Point} stability parameter and definition of stability regions that determine the stability rate.
\item Design of a feedback control system which allows the movement of the robot joints in order to maintain stability at all the times.
\end{itemize}

%Este Trabajo Fin de Máster tiene como principal objetivo realizar el control de estabilidad del robot humanoide TEO (\textit{Task Environment Operator}) utilizando sensores de Fuerza-Par y tratar los problemas y consideraciones que deben ser tenidas en cuenta cuando se diseña el sistema de control de un robot humanoide.

%Los objetivos propuestos son:
%\begin{itemize}
%\item Puesta en marcha y adquisición de datos en tiempo real de los sensores de Fuerza-Par acoplados a los tobillos del robot, que serán la principal fuente de realimentación del lazo de control.
%\item Cálculo de los parámetros de estabilidad de interés, entre ellos el ZMP \textit{Zero-Moment Point}, y evaluar en funcion de éstos la estabilidad o no estabilidad del robot.
%\item Diseño de un sistema de control realimentado que permita rectificar los parámetros de estabilidad mencionados anteriormente, y por tanto, la postura del robot para lograr una mayor estabilidad. 
%\end{itemize}



