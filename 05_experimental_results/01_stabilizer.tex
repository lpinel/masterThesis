\section{Stabilizer}
The Stabilizer structure explained in section \ref{chap:control_sec:stabilizer} is now implemented in TEO robot throuth the ZMP LQR controller designed in section \ref{chap:control_sec:lqrDesign}. The stabilizer is designed to control the ZMP position through commanding different angulat positions to robot joint ankles.

\subsection{Position control}

\subsection{Velocity control}

\subsection{Model changes}

\subsection{ZMP areas}
As mentioned before, the robot can recover its balance or not depending on the ZMP position and which parts of its body are compensating the fall down. The ankle strategy should be enough to recover balance from a low disturbance. In order to decide this strategy limit, some experimental trials have been done. 

Firstly, the ankle strategy ZMP limit was obtained starting from an upright position -blocking arms, neck and trunk joints-, and giving increasing ZMP reference positions. After \textcolor{red}{five} tests, the average ZMP limit forwards is \textcolor{red}{¿? m} and backwards \textcolor{red}{¿? m}. The great difference of ZMP limit between forwards and backwards is mainly due to the shape of the supporting area (the sole). The mechanical design of the robot feet and legs, make that forwards there is a greater surface in contact with the ground because the center of the ankle joint is displaced rearwards of the center of the sole (see Appendix \ref{app:footDraw}).

\textcolor{red}{Quizá una gráfica de los experimentos y el limite en cada uno de ellos con la media?}

Other experiment carried out is related to the maximum torque of the ankle motors. Brushless motors installed at ankle joints are ......... The highest torque the motors can exert is the maximum angle, thus the maximum ZMP position that ankle joints themselves can reach.


\textcolor{red}{Al final, una figura con la huella del pie y las zonas delimitadas.}

