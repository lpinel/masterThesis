\section{Stabilizer}
The Stabilizer structure explained in section \textcolor{red}{4.4} is now implemented in TEO robot throuth the ZMP LQR controller designed in section \textcolor{red}{4.3}. 

Firstly the stabilizer is designed to control the ZMP position through commanding different angular positions to robot joint ankles.

\subsection{ZMP areas}
As mentioned before, the robot can recover its balance or not depending on the ZMP position and which parts of its body are compensating the fall down. Remember that the ankle strategy should be enough to recover from a low disturbance but if it increases, it should change to the hip strategy or even make a step. In order to decide when the stability strategy should change, some experiments have been done taking the robot in an increasing ZMP position until it looses balance. 

Firstly, the ankle strategy ZMP limit was obtained starting from an upright position -blocking arms, neck and trunk joints-, and giving increasing ZMP reference positions. After \textcolor{red}{five} tests, the average ZMP limit forwards is \textcolor{red}{0.74 m} and backwards \textcolor{red}{0.02 m}. The big difference of ZMP limit between forwars and backward is mainly due to the shape of the supporting area (the sole). The mechanical desing of the robot feet and legs, make that forwards there is a greater surface in contact with the ground because the center of the ankle joint is displaced rearwardly of the center of the sole (see \textcolor{red}{APPENDIX:PLANES}).

\subsection{Ankle strategy}


\subsection{Hip strategy}
