\section{ZMP computation}
Using F-T sensors located at the ankles of the robot, the ZMP values are obtained from equations \eqref{eq:xzmp} and \eqref{eq:yzmp}, where the distance between the ground and the sensor center is 194 milimeters (sensor height is obtained from \textcolor{red}{APPENDIX: planes}).

Data read form de Data Acquisition Card is scaled to SI units, clustered to a YARP Bottle object and sent through YARP ports. Each sensor rate is about $20 \mu s$, thus four sensors reading rate is about $80 \mu s$. The problem comes when the data is sent trough YARP ports and there is a client receiving and processing this data. Then the update rate hugely decreases to $10 - 50 ms$, deppending on the reader processing cycle. That occurs because the arrival of updates is delayed until the client completes processing and no updates will ever be lost on the client side \cite{Yarp2006}.
%the default YARP behaviour is to send data when the client is free. That means that data is received befor any processing is donw by the client. If updates arrive faster than processing occurs, then updates will be lost from time to time, but the most recen update received will always be available to the client inmediately when processing is completed.

For a better visual observation of the ZMP data, it has been developed a  Python User Interfase using the module \textit{Matplotlib} to represent the ZMP in the ground X-Y plane with the sole borders in both single and double support. The ZMP stability areas are also represented according to the limits obtained later in section \textcolor{red}{section} as shown in Figure \textcolor{red}{FIGURA: dos imagenes de single y double support con las areas pintadas.}