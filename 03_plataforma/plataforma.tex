
\chapter{Platform description} 
\label{cap:platform_description}
\section{Humanoid robot TEO}
Humanoid robot RH-2, also known as TEO (Task Environment Operator), from University Carlos III of Madrid, is and advanced version of humanoid robots RH-0 and RH-1. TEO is 165 cm high overtaking 150 cm of RH-1 and 120 cm of RH-0. It wheights about 60 kg and it can carry about 2 kg of payload. It has 26 DOFs (28 DOFs taking into account head motors), more DOFs than in previous versions. In Figure \ref{fig:gdl} one can see the robot DOFs, besides their direction of rotaton, being 6 DOFs for each leg, 6 DOFs for each arm, 2 DOFs for the torso and 2 DOFs for the head.

\begin{figure}[!hbt]
\centering
\includegraphics[scale=0.45]{teo_gdl.png}
\caption{Distribución de grados de libertad del robot TEO.}
\label{fig:gdl}
\end{figure}

The robot has 4 microprocessors: locomotion, manipulation, artificial vision tasks and last, the main processor which manages the others. The locomotion processor, that controls the legs and the torso, will be responsible for getting the sensors information and maintain the robot in a balance and upright position, being static or in a walking cycle. The manipulation processor controls the movement of the arms and the head. The processor responsible for the computer vision uses a camera with infrarred sensor ASUS located in the head.

The communication system is based on the CAN-bus protocol. Making a sagittal and transversal division, there are 4 CAN-bus lines: 1 line per each arm and neck, and 1 line per each leg and torso.

For data acquisition, the robot has a inertial sensor located at the trunk and Force-Torque sensors located at the robot ankles and wrists. F-T sensors are plugged into real time data acquisition PCI cards. Including F-T sensors is an important difference and advantage respect RH-2 predecessors. They allow to close the control loop and then, to obtain a kind of feedback which is necessary to accomplish tasks successfully.

The Control level is divided into 2 layers. At joint level, each servo not only closes the servo loop, it also synchronizes with other devices and absolute encoders provide a feedback of joint motors. At task-oriented control level, high level executions are done. This layer is directly connected with the joint level control. Examples of appliance are manipulation of objects or stability control which is the aim of this project. Figure \ref{fig:comms} shows the general communication diagram.

\begin{figure}[!hbt]
\centering
\includegraphics[scale=0.5]{comms.pdf}
\caption{Basic communications diagram}
\label{fig:comms}
\end{figure}

\section{Force/Torque sensors}
Force-Torque (F-T) sensors are based on strain gauge sensors arranged in such a way that allows to obtain force and moment measures in all axes of the 3D space. 

The sensors used in the platform, are the commercial JR3 F-T sensors described in Table\ref{table:sensores}. Look at the full scales difference between the sensors used in the wrists joints and the ones used in the ankle joints. Ankle sensors must be able to support greater forces and moments including the ones exerted by the own robot.

\begin{table}[!hbt]
\centering
\begin{tabular}{|c|c|c|c|c|}
\hline
Joint & Model & $F_{x,y}$ & $F_z$ & $M_{x,y,z}$\\
\hline
Wrist & 50M31A & 100N & 200N & 5 Nm\\ 
\hline
Ankle & 85M35A & 250N & 500N & 212Nm\\
\hline
\end{tabular}
\caption{F-T sensor models and characteristics. [JR3 Inc.]}
\label{table:sensores}
\end{table}

According to the manufacturer, the two first digits of the model show the sensor diameter, followed by the serie, and the next two digits, the thickness. As mentioned before, the ankle sensors are bigger and they support greater forces and moments. The sensors used in this Master Thesis are the ones mentioned in Table \ref{table:sensores} for ankle joints.

Serie M sensors include inner electronics in order to filter noise, digital output to use a data acquisition PCI card from the same manufacturer and an analogical output option. The nominal precission of all sensors of serie M is 1\% of full scale, and a 1/4000 full scale resolution.

The sensors used in this work, provide the option of acquiring data in International System Units or Imperial System Units, according to Figure \ref{fig:sensor}.

\begin{figure}[!hbt]
\centering
\includegraphics[scale=0.25]{sensor.pdf}
\caption{JR3 model 85M35A Force-Torque sensor}
\label{fig:sensor}
\end{figure}

\section{Data Acquisition}
The PCI cards used for data acquisition are PCI 1592D from JR3 Inc, which has 4 ports (named as in Figure \ref{fig:daq}). The sensors are plugged through a 6 or 8 pinout cables (RJ-11 and RJ-45, respectively). In the case of RJ-45, two pinouts are not used. The PCI card uses these cables to receive high speed data and provide power supply to the sensors. About the PCI supply, it is provided by the PCI slot form the computer where it is installed.


\begin{figure}[!hbt]
\centering
\includegraphics[scale=0.4]{daq.pdf}
\caption{JR3 data acquisition card}
\label{fig:daq}
\end{figure}

In order to access to received data from the sensors, it is necessary to access to card memory, specifying the memory address for each available data. These addresses can be found at Appendix \ref{app:jr3card}.

It is important to take into account that the forces and torques obtained from the sensors and processed by the card, are in the International System Units. Forces are given in Newton [N] and torques are given in tenths of a newton per meter [dN·m].

\subsection{Acquisition program}
The \textit{jr3pci4channelYarp} program reads data from the 4 F-T sensors of the robot. Data read form de Data Acquisition Card is scaled to SI units, clustered to a YARP Bottle object and sent through YARP ports. Figure \ref{fig:daqprogram} summarizes the data acquisition procedure.

\begin{figure}[!hbt]
\centering
\includegraphics[scale=0.5]{daqprogram.pdf}
\caption{Data acquisition diagram}
\label{fig:daqprogram}
\end{figure}

\textcolor{red}{Captura reading with lib ncurses}

Program cycle time is about $20 \mu s$ for each sensor, thus four sensors reading cycle time is about $80 \mu s$. The problem comes when the data is sent trough YARP ports and there is a client receiving and processing this data. Then the update rate hugely decreases to $10 - 50 ms$, deppending on the reader processing cycle. That occurs because the arrival of updates is delayed until the client completes processing and no updates will ever be lost on the client side \cite{Yarp2006}.
%the default YARP behaviour is to send data when the client is free. That means that data is received before any processing is donw by the client. If updates arrive faster than processing occurs, then updates will be lost from time to time, but the most recen update received will always be available to the client inmediately when processing is completed.

Using these Force-Torque measurements from the sensors located at the ankles of the robot, the ZMP values are obtained using equations \eqref{eq:xzmp} and \eqref{eq:yzmp}, where the distance between the ground and the sensor center is 194 milimeters (sensor height is obtained from Appendix \ref{app:sensorDraw}.

For a better visual observation of the ZMP data, it has been developed a Python User Interfase using the module \textit{Matplotlib} to represent the ZMP in the ground X-Y plane with the sole borders in both single and double support. \textcolor{red}{The ZMP stability areas are also represented according to the limits obtained later in section X} as shown in Figure \textcolor{red}{FIGURA: dos imagenes de single y double support con las areas pintadas.}